% ======================================================= %
% Document: TEMPLATE FOR RESPONSES TO REVIEWERS
% Author: Andrea Ballatore
% Date: Jan 7, 2013
% Source: https://raw.githubusercontent.com/ucd-spatial/Datasets/master/tex_response_to_reviewers_template/responses_to_reviewers.tex
% Modified by Eduard Szöcs, 10.03.2015
% ======================================================= %
\documentclass[12pt]{article}

% packages
\usepackage{xr}
\externaldocument[ms-]{BetterGrowthModeling_Ecology_Revision1}

\usepackage{graphicx}
\usepackage{url}
\usepackage[usenames,dvipsnames]{xcolor}
\usepackage{color}
\definecolor{mygray}{gray}{0.6}
\usepackage[utf8]{inputenc}
\usepackage[onehalfspacing]{setspace}
\usepackage[
	round,	%(defaultage in the main file and \input ) for round parentheses;
	colon,	% (default) to separate multiple citations with colons;
	authoryear,% (default) for author-year citations;
	sort,		% orders multiple citations into the sequence in which they
]{natbib}
\usepackage[%disable
	]{todonotes}

\usepackage{anysize}
\marginsize{2.5cm}{2.5cm}{1.5cm}{2.5cm}

\usepackage{xr}
\externaldocument[ms-]{BetterGrowthModeling_r1}

% macros
% add a counter
\newcounter{cN}
\setcounter{cN}{0}

\newcommand{\comment}[1]{
	\vspace{2em}
	\refstepcounter{cN} % incrment counter
	\noindent \hangindent=0em \textbf{\textcolor{Maroon}{\uline{Comment \thecN}:~}}\emph{``#1''}
	}

\newcommand{\response}[1]{
	\\[0.25em]
	\hangindent=2.3em \textbf{\textcolor{NavyBlue}{\uline{Response}:~}}#1
	}

\newcommand{\revise}[1]{{\color{Mahogany}{#1}}}
\newcommand{\spe}[1]{{\color{red}{#1}}}

\usepackage[normalem]{ulem}
\definecolor{darkred}{rgb}{1,.6,.6}
\DeclareRobustCommand\problemline{\bgroup\markoverwith{\textcolor{darkred}{\rule[-0.9ex]{4pt}{3pt}}}\ULon}
\DeclareRobustCommand{\problem}[1]{\problemline{#1}} % soul
\setcounter{secnumdepth}{-1}

\begin{document}
% ======================================================= %
\title{Manuscript ECY24-0612 --- Response to reviewers}

\maketitle
% ======================================================= %
\noindent To the editorial board,

Thank you for the opportunity to submit a revision of our manuscript to \emph{Ecology}. In response to Reviewer 1, we have re-analyzed all IPM case studies to include estimates of uncertainty in model outputs. Our new results are summarized in the revised Table 1 and discussed throughout the text. We have also incorporated the questions and suggestions raised by Reviewers 2-3. We describe these and other changes in greater detail below, where we reproduce comments from the associate editor and reviewers and provide our point-by-point responses. 

All of our changes are denoted in the revised manuscript with \revise{Mahogany font}. 
We think the review process has greatly strengthened our work such that it is now suitable for publication.
We hope you agree. 

\vspace{2em}
Sincerely,

Tom Miller

Steve Ellner

\newpage

% ======================================================= %
\section{Response to Reviewer 1}
\vspace{-2em}

\comment{Miller and Ellner present a well written and useful guide to assessing and modeling non-gaussian growth in developing integral projection models. As the authors point out, calls for non-gaussian growth models for IPMs have been common, but the tools to do this have often been less clear. The authors produce a well-organized and clearly written recipe book to begin to tackle this challenge. I think the manuscript will absolutely be useful to folks that work in this field and will move discussion on this topic forward. I think it’s very worthy of publication once my questions below regarding uncertainty have been addressed.}
\response{We appreciate this reviewer's supportive comments and constructive suggestions.}

\comment{My primary concern, which should be easily addressable in a revision, relates to the lack of uncertainty quantification in any of the IPM outputs. The authors demonstrate differences in the two models (Table 1, Fig 7, etc.), but use only point predictions, not accounting for demographic parameter or other uncertainty that are buried in the models. My strong suspicion (although I would be happy to be proven wrong) is that there would be overlap in the inference derived from the gaussian and non-gaussian models, even in the lichen and creosote case study, after accounting for uncertainty. Especially since the improved model contains several more parameters to estimate each of which have uncertainty. The authors point to Bayesian methods as a natural way incorporate this uncertainty, but uncertainty quantification and propagation is an essential part of any statistical modeling that needs to be included regardless of the approach. I think of uncertainty quantification is part of the “getting it right” the authors speak to on line 412.}
\response{We completely agree with the reviewer and we appreciate being pushed to practice what we have preached. In response, we have re-analyzed all IPM case studies such that we now provide bias-corrected bootstrap confidence intervals on all model outputs, in addition to point estimates. These results are summarized in the updated Table 1. The reviewer's intuition was correct that the differences between Gaussian and ``improved'' growth models were small relative to their respective uncertainties, and their confidence intervals were strongly overlapping for most results. We address this point at several places throughout the text (lines \ref{ms-cactus.uncertainty}, \ref{}, \ref{}). 
%Our paper never hinged upon ``significant'' differences due to better growth modeling, so while the uncertainty estimates add %valuable nuance they do not change the message of the paper. \spe{The reviewer's intuition,
\spe{Interestingly, the largest differences between the Gaussian and improved growth models were differences in the
size and location of the confidence intervals --- the more realistic model lead to different uncertainty quantification. 
The reviewer's intuition, like ours, was that the additional parameters in the more realistic models would increase confidence interval widths, but that was not always actually the case. } }

\comment{I also think there is a larger philosophical question buried in here that the authors may also be able to begin to address. A question that is part of a larger discussion that the users of structured population models (which I consider myself part of) need to have.  Do we need models that attempt to capture more aspects of biological reality or do we just need to be more realistic in how we quantify and propagate uncertainty? The whole purpose of including uncertainty is to acknowledge that we recognize that any model will be wrong.}
\response{Do we need models that capture biological reality? Yes. Do we need to be more realistic in how we quantify and propagate uncertainty? Also yes. We don't see tension between these priorities \spe{(because omitting a relevant feature of biological reality will likely distort uncertainty quantification)}
although we acknowledge (or agree, if this is the reviewer's point) that our field has been more preoccupied with the former than the latter, on average. These philosophical questions are important and fun to think about, but we do not see this paper as the best place to address them. We want to keep the reader focused on practical aspects growth modeling.}

\comment{A final general note.  It’s worth keeping in mind, even when there are differences in predictions between the two approaches, we don’t know which one is producing more accurate predictions of things like life history characteristics (lifespan, generation time, etc). Models that best match observed data do not always provide the most accurate predictions (bias-variance tradeoff), so just seeing differences is not necessarily evidence of improvement.  }
\response{We agree with this general point, and we have added it to our discussion (line \ref{}). }

\comment{193: I am not 100\% convinced that having a single model for skewed growth is always advantageous, rather than a mixture model. In some instances, particularly forecasting, mixture model approaches such as described in Shriver et al. 2012 may have some advantage over the authors proposed approach. If there are different processes leading to “normal growth and shrinkage” vs. “extreme shrinkage” these processes may drive by different factors and not always covary. Thus, the most biologically accurate model would account for these processes separately }
\response{We agree - if the aggregate distribution masks multiple underlying processes, mixture models might be preferable. We see this as an empirical question: can a single model (or process) explain the patterns in the data, or is a more complex model needed? In the case of the lichen case study (modeled by Shriver et al. 2012 with a mixture of ``normal shrinkage'' and ``extreme shrinkage''), we were satisfied that the Johnson's $S_{U}$ distribution with size-dependent skewness adequately captures both types of shrinkage (see Figure 5). However, for cases were a single distribution might fail, we have added the idea of mixture growth distributions to the Discussion (line \ref{}).}

\comment{Table 1: I think these are interesting differences, but I don’t think the authors can claim they are as meaningful without including estimates of uncertainty. At the very least the inference of the life history parameters in Table 1 should account parameter uncertainty from the demographic rates estimated from data. }
\response{As described above, Table 1 now provide bias-corrected bootstrap confidence intervals to represent uncertainty in model estimates.}

\comment{308: Claims of over- or under-estimation in this case have to be relative to the other model, the truth is not known. Also similar to table 1, I find it very challenging to know if this is really a meaningful difference without uncertainty quantified.}
\response{We have re-phrased this sentence to make it clear that we are describing differences between models and not between a model and the truth (line \ref{}). While this text is describing a slightly different analysis (of the stochastic growth rate $\lambda_S$) from the contents of Table 1, we think the new information in the table makes it clear that effects of improved growth modeling were small relative to uncertainty in model outputs and we reinforce this point in the Discussion (line \ref{}).} 

\comment{412: One could argue we will never get it exactly right, that is why uncertainty quantification is so important. }
\response{We appreciate the reviewer's healthy and appropriate emphasis on uncertainty, but we think it is important to recognize the distinction between parameter uncertainty and model uncertainty. The reviewer's comments, here and throughout their review, are about parameter uncertainty (e.g., given the data and the model, how confidently can we estimate $\lambda$)? Our sentence about ``getting it right'', and indeed our entire paper, are essentially about model uncertainty (e.g., are the data more consistent with a Gaussian growth model or some other growth model?). While we will always have parameter uncertainty (and it's always important to quantify), we have found that most demographic data sets can yield high certainty about models that are or are not consistent with real size transitions -- if one makes the effort to check. This is what we mean by ``getting it right''.}

\comment{Supplement: 659: I try not to be dogmatic in my statistical thinking, but I do find myself thinking throughout this that Bayesian estimation would not just be an alternative, but perhaps simpler.}
\response{Yes and no. We agree that Bayesian estimating often makes uncertainty quantification simpler, at least for those already familiar with it. In practice, however, the widely used Bayesian software packages (Stan, JAGS, WinBUGS) do not currently include built-in functions for most of the non-Gaussian growth distributions that we discuss. Custom distributions can be coded in any of those languages, \spe{and random walk Metropolis-Hastings can be coded for any likelihood (if you can also wait patiently for enough independent draws from the posterior)}. But either of those is a fairly high technical barrier for the average population ecologist. This is the main reason we do not advocate more strongly for Bayesian approaches, and we now include this rationale in our discussion (line \ref{}). \spe{An additional complication is that the various components of an IPM are often parameterized
separately from partially distinct and partially overlapping subsets of the overall data set. Drawing indpendently from the
posterior parameter distributions for the survival, growth, fecundity, offspring size, etc. models then may not not accurately represent the correlated uncertainty in the parameters of the different models, and this would affect the uncertainty about quantities derived from the model as a whole such as $\lambda$.}  }

\section{Response to Reviewers 2 and 3}
\vspace{-2em}

\comment{Overall, we genuinely enjoyed reading the paper. It is a good fit for the journal and article type, and we are very confident it will be valued by everyone who works with IPMs.
\\
\\
Minor comments
Though we think the manuscript is more or less publishable in its current state, we have a few minor suggestions that might improve it.
}
\response{We appreciate the encouraging feedback and constructive suggestions from these reviewers. Thank you.}

\comment{Given the authors’ experience, could they speculate about what kind of IPM metrics are more likely to be impacted by adopting the wrong growth model? Our intuition is that population growth metrics (PGR, R0) are likely reasonably robust to growth kernel assumptions, whereas lifecycle metrics will be more sensitive to those assumptions.}
\response{There were no clear patterns that emerged from our six case studies regarding IPM metrics that were more or less sensitive to the ``wrong'' growth model. In fact, they were generally insensitive, especially in light of the uncertainty intervals that we now provide. However, we speculate that higher moments of life history traits, such as the variance or skewness of lifespan of lifetime reproductive success, might be more sensitive to the distribution of size transitions, and we have added this idea and some accompanying rationale to the Discussion (line \ref{}). We appreciate the suggestion.}

\spe{I did the higher moments for the lichens, and they were not much more sensitive. Here are the mean, SD, and skewness of lifespan, followed by mean, SD, and skewness of LRO, for the Gaussian and JSU models: }
\begin{verbatim}
Gaussian: 6.44  15.2  6.12     1.40 6.45 8.22 
JSU:      5.39  12.1  6.58     1.03 5.29 9.45 
\end{verbatim}
\spe{So JSU says that lifespan and LRO distributions should be less variable but more asymmetric, but the difference is
not very dramatic.} 


\comment{In the sentence beginning on line 287 “Through repeated trial and error…”, it would be helpful to see that unpacked a little bit. i.e., what does the error part of that process look like in this context? Being shown what does not work is as useful as seeing what works. }
\response{This is a great point. The ``repeated trial and error'' we referred to was and remains available to step through in our corresponding R script (), but to the reviewers' point, there is value in unpacking this a bit in the text, which we now do (line \ref{}).}

\comment{Similarly, in the sentence beginning on line 366, “We were happier with skewed t”, it would be useful to know why the authors were happier with skewed t (i.e., by what standard was it judged to be a better fit than the alternatives).}
\response{We have expanded this text to be more transparent about why we went with the skewed $t$ for this case study (line \ref{}).}

\comment{The figures (e.g., Figure 3) are neither black and white printable nor colourblind friendly. This could be easily remedied by using a colourblind-friendly palette or using different line types to represent the two values being plotted rather than shades of red and blue with very similar values.}
\response{Thank you for pointing this out. We have re-drawn our figures with color-blind and printer-friendly palettes.}

\comment{Figure 5 doesn’t have panel letters (c.f., figures 3 and 4 for example), and there is a typo on line 337, “The interaction model with strongly favoured...” should read “...was strongly favoured...”. }
\response{Thanks for catching this -- fixed.}

\comment{In the paragraph which begins on line 47 the authors use both normal and Gaussian in the context of the distribution. Choose one and stick with it rather than using both?}
\response{Fixed, thank you.}

% ======================================================= %
\end{document}
% ======================================================= %
