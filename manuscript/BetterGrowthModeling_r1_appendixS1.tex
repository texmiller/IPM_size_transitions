\documentclass[12pt]{article}
\usepackage{amsmath,amsfonts,amsthm,amssymb,graphicx,authblk}
\usepackage[font={footnotesize,singlespacing},labelfont=bf]{caption}
\usepackage{blkarray, bm} 
\usepackage{float,afterpage}
\usepackage[running,mathlines]{lineno}

\usepackage[verbose,letterpaper,tmargin=2.54cm,bmargin=2.54cm,lmargin=2.54cm,rmargin=2.54cm]{geometry}
\usepackage[authoryear,sort]{natbib}
\usepackage[dvipsnames]{xcolor}
\usepackage{hyperref}

\usepackage[compact,small]{titlesec}
\usepackage[moderate]{savetrees}

\usepackage{enumitem}
\setlist{topsep=.1em,itemsep=-0.2em,leftmargin=0.75cm}

\setlength{\parindent}{0.35in}
% \usepackage[sc]{mathpazo} %Like Palatino with extensive math support
\usepackage[scale=1]{newtxtext,newtxmath} 

\usepackage{siunitx}

\usepackage[nodisplayskipstretch]{setspace} 

% Coloring of R code listings
\usepackage[formats]{listings}
\usepackage{color}
\definecolor{mygreen}{rgb}{0.1,0.5,0.1}
\definecolor{mygray}{rgb}{0.5,0.5,0.5}
\definecolor{mymauve}{rgb}{0.58,0,0.82}
\definecolor{mygrey}{rgb}{0.3,0.3,0.1}
\lstset{
	language=R,
	otherkeywords={data.frame},
	basicstyle=\normalsize\ttfamily, 
	commentstyle=\normalsize\ttfamily,
	keywordstyle=\normalsize\ttfamily,
	stringstyle=\color{mymauve}, 
	commentstyle=\color{mygreen},
	keywordstyle=\color{blue},
	showstringspaces=false, xleftmargin=2.5ex,
	columns=flexible,
	literate={~}{{$\sim \; \; $}}1,
	alsodigit={\.,\_},
	deletekeywords={on,by,data,R,Q,mean,var,sd,log,family,na,options,q,weights,effects,matrix,nrow,ncol,wt,fix,distance},
}
\lstset{escapeinside={(*}{*)}} 

\lstdefineformat{Rpretty}{
	; = \space,
	\, = [\ \,\]]\string\space,
	<- = [\ ]\space\string\space,
	\= = [\ ]\space\string\space}


%\usepackage{lineno}
\renewcommand{\refname}{Literature Cited}
\renewcommand{\floatpagefraction}{0.9}
\renewcommand{\topfraction}{0.99}
\renewcommand{\textfraction}{0.05}

\clubpenalty = 10000
\widowpenalty = 10000

\sloppy 

\usepackage{ifpdf}
\ifpdf
\DeclareGraphicsExtensions{.pdf,.png,.jpg}
\usepackage{epstopdf}
\else
\DeclareGraphicsExtensions{.eps}
\fi

% commands for commenting
\newcommand{\tom}[2]{{\color{red}{#1}}\footnote{\textit{\color{red}{#2}}}}
\newcommand{\steve}[2]{{\color{blue}{#1}}\footnote{\textit{\color{blue}{#2}}}}
\newcommand{\revise}[1]{{\color{Mahogany}{#1}}}
%%%%%%%%%%%%%%%%%%%%%%%%%%%%%%%%%%%%%%%%%%%%% 
%%% Just for commenting
%%%%%%%%%%%%%%%%%%%%%%%%%%%%%%%%%%%%%%%%%%%%
\usepackage[dvipsnames]{xcolor}
\newcommand{\comment}{\textcolor{blue}}
\newcommand{\new}{\textcolor{red}}

\newcommand{\be}{\begin{equation}}
	\newcommand{\ee}{\end{equation}}

\newcommand{\red}{\textcolor{red}}

\sloppy

\begin{document}


% ######################## Appendices ##############################
\newpage 
\clearpage 
\appendix



\begin{spacing}{1.35} 

	\centerline{\Large{\textbf{Appendix S1}}}
	\renewcommand{\thetable}{S-\arabic{table}}
	\renewcommand{\thefigure}{S-\arabic{figure}}
	\renewcommand{\thesection}{S.\arabic{section}}
	\renewcommand{\theequation}{S\arabic{equation}}
	\setcounter{page}{1}
	\setcounter{equation}{0}
	\setcounter{figure}{0}
	\setcounter{section}{0}
	\setcounter{table}{0}
	\section{The Jones-Pewsey (2009) sinh-arcsinh distributions}
	\citet{jones-pewsey-2009} introduced a tractable generalization of the Normal distribution with two 
	additional parameters determining asymmetry (skewness), and tail weight (kurtosis) which can be either 
	lighter or heavier than the Gaussian. The generalizatin is defined through transformation of the
	Normal distribution using the hyperbolic sine function (sinh) and its inverse (asinh), 
	as follows. The base distribution $f_{\epsilon,\delta}$  is the 
	probability density of the random variable $X_{\epsilon,\delta}$ where  
	\be
	Z = \sinh (\delta \; \mbox{asinh}(X_{\epsilon,\delta}) - \epsilon)
	\label{eqn:JP1}
	\ee
	and $Z$ has a Normal(0,1) distribution. Equivalently, 
	\be
	X_{\epsilon,\delta} = \sinh \left( \delta^{-1} \big[\mbox{asinh}(Z) + \epsilon \big] \right), \quad Z \sim \mathcal{N}(0,1).
	\label{eqn:JP2}
	\ee
	Parameters $\delta=1, \; \epsilon=0$ give the Normal(0,1) distribution. Skewness has the sign of $\epsilon$, and
	$\delta > 0$ controls tail weight, with heavier than Gaussian tails for $\delta<1$ and lighter than Gaussian tails for $\delta > 1$. We show below that the nonparametric kurtosis (eqn. 3 in the main text) depends 
	only on $\delta$, not on any of the other three parameters. 
	
	The density function for $X_{\epsilon,\delta}$ is given by \citet[][eqn. 2]{jones-pewsey-2009}, 
	\be
	\begin{aligned}
		f_{ \epsilon,\delta}(x) & = C(x) \exp\{-S(x)^2/2\} \{2\pi(1+x^2)\}^{-1/2}  \\
		\mbox{ where }  S(x) & =  \mbox{sinh}(\delta \mbox{ asinh}(x)- \epsilon), \\
		C(x)  & =  \sqrt{1 + S(x)^2} = \mbox{cosh}(\delta \mbox{ asinh}(x)- \epsilon).
	\end{aligned} 
	\ee
	The attainable combinations of skewness and kurtosis are 
	very broad compared to other families, and come very close to the theoretical limit of
	kurtosis as a function of skewness \citep[][Fig.  2]{jones-pewsey-2009}. 
	Additionally, eqn. \eqref{eqn:JP2} makes it straightforward to generate random numbers and to compute 
	the probability density, cumulative distribution, and quantile functions. 
	There are also analytic formulas for the first four non-central 
	moments \citep[][p. 764]{jones-pewsey-2009} in terms of the Bessel function $K_{\nu}$, which is
	\texttt{BesselK} in base R and \texttt{besselk} in \textsc{Matlab} and GNU \textsc{Octave}.
	
	\citet{jones-pewsey-2009} defined a four-parameter distribution with location 
	parameter $\mu$ and scale parameter $\sigma$ as the distribution of $\mu + \sigma X_{\epsilon, \delta}$, 
	which has density function 
	\be
	f_{\mu, \sigma,  \epsilon,\delta}(x)  = \sigma^{-1}f_{ \epsilon,\delta}( \sigma^{-1}(x - \mu)). 
	\label{eqn:JP4} 
	\ee
	Terminology on this distribution has become somewhat confused. In the \textbf{mgcv} R package 
	it is called \texttt{shash}, while in the \textbf{gamlss} package it is called \texttt{SHASHo2} 
	and \texttt{SHASH} is a related but different distribution. To sidestep 
	this confusion we refer to \eqref{eqn:JP4} as the JP4 distribution (``Jones-Pewsey 4 parameter''), 
	and refer to \eqref{eqn:JP2} as JP2. 
	
	As is the case for most four-parameter distribution families, 
	the location parameter $\mu$ is not the mean of the JP4 distribution, 
	and $\sigma$ is not the standard deviation (additionally, $\epsilon$ is not the skew and $\delta$ is not the kurtosis). 
	We therefore define a new four-parameter distribution family, JPS, by shifting and scaling JP2
	so that the location parameter $\mu$ is the mean, and the scale parameter $\sigma$ is the standard deviation. 
	This form can then be used in custom likelihood functions that ``import'' the fitted mean 
	and standard deviation from a Gaussian pilot model, in the same way that the skewed $t$ distribution 
	was used in our lady orchid case study. 
	
	Let $m(\epsilon,\delta)$ and $s(\epsilon, \delta)$ denote the mean and standard deviation of the 
	JP2 distribution. Then define 
	\be  
	X_{JPS} = \mu \; + \; \sigma \left(\frac{X_{\epsilon,\delta} - m(\epsilon,\delta)}{s(\epsilon,\delta)}\right).
	\ee
	The right-hand term in parentheses has mean 0 and variance 1, so 
	$X_{JPS}$ has mean $\mu$ and variance $\sigma^2$, with $\epsilon$ and $\delta$ controlling skewness
	and tail weight as in JP2. Because $\mu$ and $\sigma$ have no effect on the nonparametric kurtosis, 
	JPR retains the property that nonparametric kurtosis only depends on $\delta$, not on the other three parameters. 
	
	Omitting some algebra, $X_{JPS}$ has cumulative distribution function 
	\be
	Pr(X_{JPS} \le x)  = Pr\left(X_{\epsilon,\delta} \le  m(\epsilon,\delta) + \frac{s(\epsilon,\delta)}{\sigma}(x - \mu)\right).
	\ee
	Differentiating both sides with respect to $x$, the probability distribution function for $X_{JPS}$ is 
	\be
	f_{JPS}(x \vert \mu, \sigma, \epsilon, \delta) =  \frac{s(\epsilon,\delta)}{\sigma} 
	f_{ \epsilon,\delta}\left(m(\epsilon,\delta) + \frac{s(\epsilon,\delta)}{\sigma}(x - \mu) \right) 
	\ee
	
	Eqn. \eqref{eqn:JP2} shows that the JP2 distribution depends on $\epsilon$ only through 
	the ratio $\epsilon/\delta$, and hence the same is true for JPS. 
	We have found that this property can be problematic for parameter estimation, 
	because of the resulting ridge in the likelihood surface with constant  
	$\epsilon/\delta$. Another problem is that when $\delta$ is large, changes 
	in $\epsilon$ have little effect. 
	
	To avoid those problems, we recommend writing likelihood functions in terms of 
	skewness and kurtosis parameters $\lambda$ and $\tau$, defined by 
	$\delta = e^{-\tau}, \; \epsilon =  \delta \lambda$ in the 
	JPS distribution. We will refer to this as the JPR distribution, with probability density 
	\be
	f_{JPR}(x \vert \mu, \sigma, \lambda, \tau) = f_{JPS}(x \vert \mu, \sigma, e^{-\tau}\lambda,  e^{-\tau}). 
	\ee
	$\lambda$ can take any real value, and the distribution's skewness has the same sign as $\lambda$. 
	$\tau$ also can take any real value, with negative values giving thinner than Gaussian tails 
	and positive values giving fatter than Gaussian tails. Because $\delta$ depends only on $\tau$, 
	JPR also has the property that the nonparametric kurtosis depends only on the tail-weight parameter $\tau$. 
	
	It is still the case that the ordinary skewness and kurtosis depend on both $\lambda$ and $\tau$, but the
	``crosstalk'' is weaker than that between $\epsilon$ and $\delta$ (in particular, the tail-weight parameter
	has much less effect on the skewness).  
	As a result, we found that likelihood optimization is numerically more stable when the likelihood function is 
	written as a function of $\tau$ and $\lambda$ rather than $\delta$ and $\epsilon$. 
	
	R code for the JP2, JPS, and JPR distributions with the usual \texttt{d,p,q,r} functions are provided 
	in the script \texttt{JP\_funs.R} in our R code archive. 
	
	\emph{Proof that NP kurtosis of JP2 depends only on $\delta$}: 
	Let $Z_\alpha$ denote the $\alpha$ percentile of a standard
	Normal distribution, $X_\alpha$ the $\alpha$ percentile of $X_{\epsilon, \delta}$ and 
	$\lambda =  \epsilon/\delta$. Then from \eqref{eqn:JP2}
	we have 
	\be
	\begin{aligned}
		X_{1-\alpha} & = \mbox{sinh} \left[ \lambda  + \delta^{-1} \mbox{asinh}(Z_{1-\alpha}) \right] ,  \\
		X_{\alpha} & = \mbox{sinh} \left[ \lambda  + \delta^{-1} \mbox{asinh}(Z_{\alpha}) \right]
		= \mbox{sinh} \left[ \lambda + \delta^{-1} \mbox{asinh}(-Z_{1-\alpha}) \right] \\
		& = \mbox{sinh} \left[ \lambda - \delta^{-1} \mbox{asinh}(Z_{1-\alpha}) \right]. 
	\end{aligned}
	\ee
	Thus  
	\be
	X_{1-\alpha} - X_{\alpha} = \mbox{sinh}(\lambda + b) - \mbox{sinh}(\lambda - b)
	\ee
	where $b = \delta^{-1}\mbox{asinh}(Z_{1-\alpha})$. We can apply the subtraction formula for sinh (eqn. 4.5.42
	in \citet{abram-steg}), namely\footnote{It's also on Wikipedia, and today it's correct, but tomorrow could
		be different.}
	\be 
	\mbox{sinh } z_1 - \mbox{sinh } z_2  = 2 \mbox{cosh}\left( \frac{z_1 + z_2}{2}\right) \mbox{sinh}\left( \frac{z_1 - z_2}{2}\right), 
	\ee  
	obtaining 
	\be
	X_{1-\alpha} - X_{\alpha} =  2 \mbox{cosh}(\lambda) \mbox{sinh}(b).
	\ee  
	The value of $b$ is independent of $\epsilon$. The $\epsilon$-dependent factor $2 \mbox{cosh}(\lambda)$ 
	cancels in the numerator and denominator of the formula for nonparametric kurtosis. \scalebox{1.25}{$\square$}
	\clearpage
	\newpage 
	
	\begin{thebibliography}{62}
		\providecommand{\natexlab}[1]{#1}
		
		\bibitem[{Abramowitz \& Stegun(1970)}]{abram-steg}
		Abramowitz, M. \& Stegun, I.A. (1970) {Handbook of Mathematical Functions. 9th printing.}
		{D}over Publications, Inc., New York.
		
		\bibitem[{Jones \& Pewsey(2009)}]{jones-pewsey-2009}
		Jones, M. \& Pewsey, A. (2009) Sinh-arcsinh distributions. \emph{Biometrika}
		\textbf{96}, 761 -- 780.
		
	\end{thebibliography}

\end{spacing} 

\end{document}